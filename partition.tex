%!TEX root = root.tex
\paragraph{Motivation}
Une centaine d'étudiants de troisième année suivent le cours de
programmation fonctionnelle. Ils ont deux heures de travaux pratiques
par semaine produisant alors plusieurs centaines de réponses. Comment
l'enseignant peut-il efficacement analyser ces réponses pour
comprendre les difficultés rencontrées par ses étudiants?

L'utilisation de {\LearnOCaml}~\cite{learnocaml} est déjà d'une grande
aide car la correction automatique des exercices permet de voir les
pourcentages de réussite de chaque étudiant. Malheureusement cela ne
suffit pas toujours car environ 80\% des étudiants obtiennent tous les
points ! Une analyse plus poussée vise à déterminer \emph{comment} les
étudiants ont répondu.
%
Nous avons donc intégré {\Asak} à {\LearnOCaml} afin de classifier les
codes des étudiants. En observant les représentants des classes
obtenues et leur taille, l'enseignant peut se faire rapidement une
idée de la façon dont les élève ont répondu et ainsi identifier les
élèves qui ont fourni une réponse inattendue et qui nécessitent
peut-être plus d'attention.

\paragraph{Approche utilisée}
Partant de l'hypothèse que les tests automatiques ont été bien écrits,
nous classifions une première fois les codes selon la note qu'ils ont
obtenue. En effet, nous ne voulons jamais identifier deux codes qui
n'ont pas eu la même note, même s'ils sont similaires
syntaxiquement. Cette première passe nous permet aussi d'avoir pour
chaque classe une hypothèse, très forte, d'équivalence sémantique.
Cette hypothèse nous permet de raffiner la fonction de calcul
d'empreinte afin d'identifier encore plus de codes.

\paragraph{Amélioration de la fonction de calcul d'empreinte}

%\paragraph{La combinaison des empreintes}
Comme nous l'avons vu précédemment, la clé de hachage d'un
terme~$\LambdaCode$ est déduite des clés de hachage de ses sous-termes
\textit{combinées dans l'ordre, de gauche à droite}. Dans cette
situation, deux termes équivalents sémantiquement et qui ne diffèrent
que par l'ordre de certains sous-termes ne sont pas envoyés vers la
même empreinte. Pour résoudre ce problème, nous avons modifié notre
fonction de prise d'empreintes pour qu'elle trie les empreintes des
sous-termes avant de les combiner. Trier ou non les sous-empreintes
est un paramètre global de~{\Asak}.

%\paragraph{L'empreinte des feuilles}
Nous avons déjà soulevé le problème du calcul des empreintes des
feuilles de l'arbre en section~\ref{sec:fingerprint}. Dans ce contexte
d'équivalence sémantique, nous pouvons supposer que dans deux arbres
de même forme, les identifiants ne sont que des alias les uns des
autres. En effet, s'il existait une différence sémantique entre deux
identifiants, les deux codes n'auraient pas la même sémantique.  Nous
avons donc choisi d'associer la même empreinte à tous les
identifiants. \yrg{C'est vraiment violent!} \am{Oui, mais ça n'est pas un problème ! Les codes sont vraiment très semblables.}

\paragraph{Résultats}

Les exemples de la figure \ref{fig:hash} sont tirés d'un corpus plus
gros: celui des réponses des étudiants de troisième année à l'exercice "Implémentez la fonction~\iocaml{rev}". $154$~réponses ont
obtenu tous les points. Sur ce corpus, {\Asak} produit $9$~classes
dont voici les cardinaux et une description succinte: $95$ fois le
premier élément est mis à la fin du reste de la liste renversée
récursivement ; $41$ fois la réponse introduit une fonction auxiliaire
récursive terminale ; $5$ fois la réponse utilise une fonction
auxiliaire non-locale ; $4$ fois le résultat de l'appel récursif est
stocké dans une variable locale et ajoute le premier élément à la fin
; $3$ fois la réponse utilise~\iocaml{List.rev}; $3$ fois la réponse
s'appuie sur~\iocaml{List.fold_left} ; $2$ fois la réponse s'appuie
sur~\iocaml{List.fold_right} ; $1$ fois la réponse utilise
un \iocaml{if-then-else} au lieu du filtrage par motifs.

L'enseignant peut déduire plusieurs choses de ce rapport. D'abord que
la plupart des étudiants n'ont pas encore acquis le réflexe d'écrire
des fonctions sur les listes dont la récursion est en position
terminale. Il peut aussi constater qu'une minorité d'étudiants
n'arrive pas encore à se détacher du paradigme impératif ou à utiliser
l'analyse de motifs plutôt qu'une expression conditionnelle. Enfin,
l'enseignant devra aussi revenir sur son code de correction automatique
qui ne capture pas le cas de triche -- pourtant grossier -- où l'étudiant
esquive l'exercice en réutilisant la fonction~\iocaml{List.rev}.

Pour des raisons de protection des données de nos étudiants, nous ne
pouvons pas publier le jeu de données qui nous a servi à produire ces
résultats. Des résultats similaires sont néanmoins reproductibles avec
la version de développement de {\LearnOCaml}~\cite{learnocaml}. Dans le
mode ``enseignant'', il suffit de faire un clic du milieu sur le nom
d'un exercice puis de spécifier la fonction à analyser pour obtenir le
partitionnement des copies produit par {\Asak}.
