%!TEX root = root.tex

Nous programmons souvent en utilisant des constructions semblant
similaires mais qui sont belles et bien des constructions sémantiquement
différentes. Cependant {\Asak} est par définition très sensible à la
forme de l'arbre {\LambdaCode} et ce "presque-sucre syntaxique" fait
différer les arbres de codes pourtant très proches. Il en résulte
qu'{\Asak} n'est \emph{pas} adapté à la détection de plagiat. Voici deux illustrations.

\paragraph{L'ordre des définitions}

L'ordre des définitions locales dans une fonction parait anodin mais il influe grandement le calcul d'empreinte puisque celui-ci est effectué récursivement. Il serait alors simple pour quelqu'un désirant fausser les résultats de réorganiser le code afin que sa sémantique reste préservée mais les arbres {\LambdaCode} produit engendrent des empreintes différentes.

\paragraph{L'inlining}

Nous remplaçons souvent mentalement les variables par leur
définition. Il s'agit cependant d'une opération complexe qui affecte
souvent la sémantique du programme et que le compilateur se risque
rarement à faire. Il en résulte que deux codes ayant exactement la
même sémantique, l'un ayant une multitude de définitions factorisées
et l'autre non engendrent des arbres très différents et donc des empreintes différentes.
