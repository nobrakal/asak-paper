%!TEX root = root.tex

Nous programmons souvent en utilisant des constructions semblant
similaires mais qui sont bel et bien des constructions sémantiquement
différentes. Cependant {\Asak} est par définition très sensible à la
forme de l'arbre {\LambdaCode} et ce "presque-sucre syntaxique" fait
différer les arbres de codes pourtant très proches. Il en résulte
qu'{\Asak} n'est \emph{pas} adapté à la détection de plagiat.
\yrg{Je ne suis pas sûr de te suivre. Tu as un exemple concret en tête?}
\am{Typiquement, l'inlining: \iocaml{let a = 1 in let b = 2 in a + b} forme un arbre très différent de \iocaml{a+b}}

\paragraph{Les définitions mutuellement récursives}

Il est par exemple très courant de définir des fonctions mutuellement
récursives alors qu'elles ne le sont pas. Le compilateur n'étant pas
capable de différencier ces cas, les arbres engendrés seront très
différents selon si des définitions successives ont étés faites via
\verb|let| ou \verb|let rec|. Heureusement dans la plupart des cas le partitionnement permet de retrouver qu'il existe de grandes similarités entre les codes.

\paragraph{L'inlining}

Nous remplaçons souvent mentalement les variables par leur
définition. Il s'agit cependant d'une opération complexe qui affecte
souvent la sémantique du programme et que le compilateur se risque
rarement à faire. Il en résulte que deux codes ayant exactement la
même sémantique, l'un ayant une multitude de définitions factorisées
et l'autre non engendrent des arbres très différents.
