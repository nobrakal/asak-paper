Dans cet article, nous avons présenté l'approche suivie par~{\Asak} pour
la détection de clones de programmes~{\OCaml} ainsi que des résultats
préliminaires concernant l'application de cet outil à {\LearnOCaml}
et à l'analyse du corpus de paquets {\Opam}.
%
Pour pallier aux limitations que nous avons explicitées, nous allons continuer
à améliorer la prise d'empreintes pour la rendre plus robuste à certaines
transformations syntaxiques locales qui préservent la sémantique.
%
Enfin, en intégrant {\Asak} a un outil comme {\Merlin}, nous allons proposer
au programmeur un outil pour éviter d'introduire de la redondance dans
les programmes~{\OCaml}.

Pour finir, les auteurs remercient la Fondation OCaml et ses sponsors pour
avoir rendu possible le stage de licence d'Alexandre Moine sans lequel
{\Asak} n'aurait pas pu voir le jour.
