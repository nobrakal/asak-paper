%!TEX root = root.tex

Dans cette section, nous formalisons la prise d'empreintes des
définitions. Dans la section~\ref{sec:lambda}, nous donnons la syntaxe
du langage~$\LambdaCode$ ainsi que les deux normalisations que nous
appliquons à ses termes. Dans la section~\ref{sec:fingerprint},
nous donnons la définition précise de notre fonction de prise
d'empreintes.

\subsection{$\LambdaCode$ normalisé}
\label{sec:lambda}

\subsubsection{Présentation de $\LambdaCode$}

\begin{figure}

\[\small
\begin{array}{rclr}
\term
  & ::= & \var & \legend{Variable} \\
  & |   & \const & \legend{Constante} \\
  & |   & \apply\term{\many\term} & \legend{Application} \\
  & |   & \lam{\many\var}\term & \legend{Abstraction} \\
  & |   & \tlet\binding\term & \legend{Définition locale} \\
  & |   & \tletrec{\many\binding}\term & \legend{Définitions récursives} \\
  & |   & \apply\primitive{\many\term} & \legend{Appel d'une primitive} \\
  & |   & \switch\term{\many\branch}{\many\branch}{\term}
        & \legend{Branchement} \\
  & |   & \staticraise\nat{\many{\term}} & \legend{Saut local} \\
  & |   & \staticcatch{\term}\nat{\many{\var}}{\term} & \legend{Expression étiquetée} \\
  & |   & \trywith{\term}{x}{\term} & \legend{Expression étiquetée} \\
  & |   & \tifthenelse\term\term\term & \legend{Branchement conditionnel} \\
  & |   & \term ; \term & \legend{Séquencement} \\
  & |   & \twhile\term\term & \legend{Boucle non bornée} \\
  & |   & \tfor\var\term\term\term & \legend{Boucle bornée ascendante} \\
  & |   & \tfordown\var\term\term\term & \legend{Boucle bornée descendante} \\
  & |   & \tassign\var\term & \legend{Affectation} \\
  & |   & \tsend\term\term{\many\term} & \legend{Appel de méthode} \\
\binding & ::= & \var = \term & \legend{Définition} \\
\branch  & ::= & n \rightarrow \term & \legend{Branche} \\
\end{array}
\]
%% Les deux constructeurs suivants doivent être ignorés.
%%   | Levent of lambda * lambda_event
%%   | Lifused of Ident.t * lambda
\caption{La syntaxe de {\LambdaCode}
$\textrm{avec }
\nat \in \mathbb{N},
\var \in \mathcal{V},
\const \in \mathcal{C},
\primitive \in \mathcal{P}$.
}
\label{fig:lambda-syntax}
\end{figure}


La syntaxe du langage $\LambdaCode$ est donnée dans la
figure~\ref{fig:lambda-syntax}. Il s'agit d'un $\lambda$-calcul avec
quelques spécificités par rapport à celui que l'on trouve dans les
présentations à saveur plus théoriques. Tout d'abord, les fonctions ne
sont pas unaires : elles ont une arité potentiellement supérieure
à~$1$. Ensuite, on distingue les primitives des constantes : les
primitives sont nécessairement appliquées. Le langage contient un
fragment impératif permettant d'affecter des variables, d'itérer
\textit{via} des boucles \iocaml{for} et \iocaml{while}, et enfin de
détourner le flot du contrôle \textit{via} les différents mécanismes
de lancement et rattrapage d'exceptions. L'appel de méthode est la
construction qui nous rappelle qu'{\OCaml} est un aussi un langage à
objets. Pour finir, $\LambdaCode$ n'a pas d'analyse de motifs mais est
muni d'un branchement n-aire introduit par le mot-clé~\iocaml{switch}.

Par manque de place, nous ne donnons pas les règles de sémantique de
ce langage et nous laissons au lecteur le soin de réfléchir à ces
dernières. Par contre, pour se convaincre que l'on ne perd pas trop de
structure en calculant la redondance sur des termes de {\LambdaCode}
et non des termes {\OCaml}, il faut prendre le temps d'expliquer les
différences entre ces deux langages.

\paragraph{Absence de types}
{\LambdaCode} est un langage non typé. On n'y retrouve donc aucune
déclaration de types. Cette absence limite donc d'emblée le champ
d'application de {\Asak}: nous ne pouvons pas détecter de déclarations
de type redondantes. Cependant, cette limitation a un avantage :
lorsque l'on oublie les types, on se donne la possibilité de détecter
plus de redondances entre des programmes de types distincts mais
partageant la même structure. En revenant aux définitions des types
qui interviennent dans deux programmes ayant la même structure, on
peut espérer détecter indirectement des redondances entre les
définitions de types.
%
Un exemple d'une telle situation sera présentée
et discutée dans la section~\ref{sec:} \yrg{Donner l'endroit où on
discute de la fonction ``isomorphe'' à Option.map}.

\paragraph{Absence de modules et de classes}
Les constructions de seconde classe (comme les définitions de modules
et de classes d'objets) ont disparu dans le programme traduit
{\LambdaCode}. Cela limite notre capacité à détecter des modules
similaires ou des classes similaires. Comme pour les définitions de
type, nous pensons qu'en nous focalisant uniquement sur les aspects
calculatoires, nous pouvons détecter \textit{a posteriori} des
définitions de modules ou de classes similaires parce qu'elles
partagent des définitions similaires.

\paragraph{Absence d'analyse de motifs}
L'analyse de motifs d'{\OCaml} a été compilée en {\LambdaCode} sous la
forme d'arbres de décision, exprimés à l'aide d'expressions
conditionnelles et de branchements. Deux analyses de motifs distinctes
syntaxiquement en {\OCaml} peuvent être envoyées vers le même arbre de
décision et donc le même code~\LambdaCode: c'est par exemple le cas de deux
analyses à~$3$ branches sur le type \iocaml{color = Red | Black |
  White} car quelque soit l'ordre des branches de
l'analyse\footnote{En supposant qu'il n'y a pas de clause
  \iocaml{when} en jeu.}, celles-ci vont être traduites vers des
branchements à $3$ cas où l'ordre des cas correspond à l'ordre des
constructeurs de données dans la définition du type.

\subsubsection{$\alpha$-renommage}

Le nom des variables est pris en compte lors de la prise d'empreintes.
Naturellement, nous avons choisi de nous abstraire de ces noms en
détectant la redondance $\alpha$-équivalence près. Il est donc
important de s'assurer que les noms de variables liées (par exemple
lors d'une définition de fonction) ne jouent aucun rôle. Pour cette
raison, nous renommons tous les identificateurs des variables liées en
des identificateurs canoniques qui codent leurs indices de De Bruijn.

\subsubsection{Réduction des définitions locales inoffensives}

En {\OCaml}, et contrairement à {\Coq}, on ne peut pas facilement
comparer les termes modulo l'évaluation : en effet, l'évaluation d'un
terme peut diverger ou produire des effets de bord incontrôlés.
%
Par contre, le compilateur sait détecter des définitions locales
très simples dont l'expansion ne pose pas de problème. Ces définitions
sont marquées par une annotation :

\begin{ocaml}
type let_kind = Strict | Alias | StrictOpt | Variable
\end{ocaml}

\noindent \iocaml{Strict} signifie que la définition peut faire
des effets de bord et ne doit pas être réduite (sauf s'il s'agit
d'une variable ou d'une constante) ; \iocaml{Alias} signifie que
la définition est pure et peut donc être réduite ; \iocaml{StrictOpt}
signifie que la définition dépend de la mémoire et ne peut donc pas
être réduite ; \iocaml{Variable} signifie que la définition sera
masquée dans la suite. \yrg{Le cas Variable me semble un peu étrange
et je ne suis pas sûr de comprendre à quoi il sert...}

\yrg{Appliquer ces expansions peut changer la complexité des programmes.
Est-il raisonnable de toutes les appliquer? Pourquoi l'expansion est
une bonne idée?}

\yrg{Section à terminer}

\subsection{Définition de la prise d'empreintes}
\label{sec:fingerprint}

Nous présentons maintenant précisément la notion d'empreinte que nous
avons choisie et la fonction qui la calcule pour tout
terme~$\LambdaCode$ (normalisé par les réécritures décrites dans les
deux sections précédentes). Comme nous l'expliquerons dans la
section~\ref{sec:related-work}, notre méthode est une variante d'un
calcul d'empreinte de la littérature~\cite{chilowicz:hal-00627811}.

L'empreinte d'un arbre de syntaxe doit témoigner de la structure de
cet arbre ainsi que l'ensemble des sous-arbres qui la constitue. Pour
des raisons d'efficacité, nous réduisons chaque sous-arbre à un entier
correspondant à son image par une fonction de hachage. Par ailleurs,
plus un sous-arbre a un grand nombre de n{\oe}uds et plus sa
contribution à l'arbre global est important : nous pondérons donc
la clé de hachage par ce nombre de n{\oe}uds.

\begin{defn}
On appelle $f$-\textit{glyphe d'un arbre de syntaxe}~$t$, le
couple~$H_f(t)$ formé d'un entier 63 bits correspondant à son nombre
de n{\oe}uds et d'un entier 128-bits correspondant à l'image de cet
arbre à travers une fonction de hachage~$f$ donnée. Pour un glype~$g$
donné, on note son poids~$w(g)$ et sa clé de hachage~$h(g)$.
\end{defn}

La définition des empreintes se dérive à partir de cette notion de glyphe
comme suit.

\begin{defn}
On appelle \textit{$f$-empreinte d'un arbre de syntaxe}~$t$, le
multi-ensemble~$E_f(t)$ formé des $f$-glyphes de ses sous-arbres.
\end{defn}

Il reste maintenant à décider quelle fonction de hachage utiliser.  En
nous appuyant sur les conclusions de~Chilowicz et ses
coauteurs~\cite{chilowicz:hal-00627811}, nous avons décidé d'utiliser
intensivement la fonction de hachage cryptographique MD5. Pour chaque
n{\oe}ud de l'arbres, on calcule une clé de hachage qui s'appuie sur
le nom de la construction utilisée et, pour procéder
incrémentalement~\cite{DBLP:conf/ml/FilliatreC06}, sur les clés de
hachage des sous-arbres.

Dans un programme, la valeur d'un litéral est très souvent liée au
contexte d'application. Du point de vue de la recherche de redondance, nous
estimons que ces valeurs constituent une forme de bruit à ignorer.
Pour que la valeur exacte des litéraux n'influence pas notre recherche
de redondance, nous avons donc décidé de ne pas prendre en compte leurs
valeurs dans le calcul de la clé de hachage.
