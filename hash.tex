%!TEX root = root.tex

Dans cette section, nous formalisons la prise d'empreintes des
définitions. Dans la section~\ref{sec:lambda}, nous donnons la syntaxe
du langage~$\LambdaCode$ ainsi que les deux normalisations que nous
appliquons à ses termes. Dans la section~\ref{sec:fingerprint},
nous donnons la définition précise de notre fonction de prise
d'empreintes.

\subsection{$\LambdaCode$ normalisé}
\label{sec:lambda}

\subsubsection{Présentation de $\LambdaCode$}

\begin{figure}

\[\small
\begin{array}{rclr}
\term
  & ::= & \var & \legend{Variable} \\
  & |   & \const & \legend{Constante} \\
  & |   & \apply\term{\many\term} & \legend{Application} \\
  & |   & \lam{\many\var}\term & \legend{Abstraction} \\
  & |   & \tlet\binding\term & \legend{Définition locale} \\
  & |   & \tletrec{\many\binding}\term & \legend{Définitions récursives} \\
  & |   & \apply\primitive{\many\term} & \legend{Appel d'une primitive} \\
  & |   & \switch\term{\many\branch}{\many\branch}{\term}
        & \legend{Branchement} \\
  & |   & \staticraise\nat{\many{\term}} & \legend{Saut local} \\
  & |   & \staticcatch{\term}\nat{\many{\var}}{\term} & \legend{Expression étiquetée} \\
  & |   & \trywith{\term}{x}{\term} & \legend{Expression étiquetée} \\
  & |   & \tifthenelse\term\term\term & \legend{Branchement conditionnel} \\
  & |   & \term ; \term & \legend{Séquencement} \\
  & |   & \twhile\term\term & \legend{Boucle non bornée} \\
  & |   & \tfor\var\term\term\term & \legend{Boucle bornée ascendante} \\
  & |   & \tfordown\var\term\term\term & \legend{Boucle bornée descendante} \\
  & |   & \tassign\var\term & \legend{Affectation} \\
  & |   & \tsend\term\term{\many\term} & \legend{Appel de méthode} \\
\binding & ::= & \var = \term & \legend{Définition} \\
\branch  & ::= & n \rightarrow \term & \legend{Branche} \\
\end{array}
\]
%% Les deux constructeurs suivants doivent être ignorés.
%%   | Levent of lambda * lambda_event
%%   | Lifused of Ident.t * lambda
\caption{La syntaxe de {\LambdaCode}
$\textrm{avec }
\nat \in \mathbb{N},
\var \in \mathcal{V},
\const \in \mathcal{C},
\primitive \in \mathcal{P}$.
}
\label{fig:lambda-syntax}
\end{figure}


La syntaxe du langage $\LambdaCode$ est donnée dans la
figure~\ref{fig:lambda-syntax}. Il s'agit d'un $\lambda$-calcul avec
quelques spécificités par rapport à celui que l'on trouve dans les
présentations à saveur plus théoriques. Tout d'abord, les fonctions ne
sont pas unaires : elles ont une arité potentiellement supérieure
à~$1$. Ensuite, on distingue les primitives des constantes : les
primitives sont nécessairement appliquées. Le langage contient un
fragment impératif permettant d'affecter des variables, d'itérer
\textit{via} des boucles \iocaml{for} et \iocaml{while}, et enfin de
détourner le flot du contrôle \textit{via} les différents mécanismes
de lancement et rattrapage d'exceptions. L'appel de méthode est la
construction qui nous rappelle qu'{\OCaml} est un aussi un langage à
objets. Pour finir, $\LambdaCode$ n'a pas d'analyse de motifs mais est
muni d'un branchement n-aire introduit par le mot-clé~\iocaml{switch}.

Par manque de place, nous ne donnons pas les règles de sémantique de
ce langage et nous laissons au lecteur le soin de réfléchir à ces
dernières. Par contre, pour se convaincre que l'on ne perd pas trop de
structure en calculant la redondance sur des termes de {\LambdaCode}
et non des termes {\OCaml}, il faut prendre le temps d'expliquer les
différences entre ces deux langages.

\paragraph{Absence de types}
{\LambdaCode} est un langage non typé. On n'y retrouve donc aucune
déclaration de types. Cette absence limite donc d'emblée le champ
d'application de {\Asak}: nous ne pouvons pas détecter de déclarations
de type redondantes. Cependant, cette limitation a un avantage :
lorsque l'on oublie les types, on se donne la possibilité de détecter
plus de redondances entre des programmes de types distincts mais
partageant la même structure. En revenant aux définitions des types
qui interviennent dans deux programmes ayant la même structure, on
peut espérer détecter indirectement des redondances entre les
définitions de types.
%
Un exemple d'une telle situation sera présentée
et discutée dans la section~\ref{sec:} \yrg{Donner l'endroit où on
discute de la fonction ``isomorphe'' à Option.map}.

\paragraph{Absence de modules et de classes}
Les constructions de seconde classe (comme les définitions de modules
et de classes d'objets) ont disparu dans le programme traduit
{\LambdaCode}. Cela limite notre capacité à détecter des modules
similaires ou des classes similaires. Comme pour les définitions de
type, nous pensons qu'en nous focalisant uniquement sur les aspects
calculatoires, nous pouvons détecter \textit{a posteriori} des
définitions de modules ou de classes similaires parce qu'elles
partagent des définitions similaires.

\paragraph{Absence d'analyse de motifs}
L'analyse de motifs d'{\OCaml} a été compilée en {\LambdaCode} sous la
forme d'arbres de décision, exprimés à l'aide d'expressions
conditionnelles et de branchements. Deux analyses de motifs distinctes
syntaxiquement en {\OCaml} peuvent être envoyées vers le même arbre de
décision et le même code~\LambdaCode: c'est par exemple le cas de deux
analyses à~$3$ branches sur le type \iocaml{color = Red | Black |
  White} car quelque soit l'ordre des branches de
l'analyse\footnote{En supposant qu'il n'y a pas de clause
  \iocaml{when} en jeu.}, celles-ci vont être traduites vers des
branchements à $3$ cas où l'ordre des cas correspond à l'ordre des
constructeurs de données dans la définition du type.

\subsubsection{$\alpha$-renommage}

Le nom des variables sera pris en compte lors du hash de la
structure. Il est donc important de s'assurer que les noms de
variables liées (par exemple lors d'une définition de fonction) ne
jouent aucun rôle.  C'est pourquoi nous faisons une itération sur
chaque arbre Lambda afin de renommer les variables liées selon leur
position.

\subsubsection{Inlining agressif}

L'arbre Lambda vient avec des informations sur les définitions
\verb|let| et notamment si elles peuvent être inliné (c'est-à-dire
remplacée par leur définition).

Nous inlinons donc toutes les définitions possibles.

\subsection{Formalisation}
\label{sec:fingerprint}

\paragraph{Notation}
On notera:

\begin{itemize}
	\item $H = \mathbb{N} \times 2^{128}$. Cela permet de représenter les couples poids/hash.
	\item $List_n(X)$ les listes de taille $n \in \mathbb{N}$ d'éléments de $X$
	\item $List_*(X) = \cup_{n\in\mathbb{N}}\ List_n(X)$
\end{itemize}

\subsection{La fonction de hash}

Nous suivons ici la méthodologie donnée par Chilowicz et
al. \cite{chilowicz:hal-00627811}: nous allons définir récursivement
une fonction de hash pour les arbres Lambda qui renverra aussi les
hashs des sous-arbre ainsi que leurs poids.

La fonction de hash est donc de type $hash : Lambda \to List_*(H)$.

%todo Pourquoi ?
Les hashs "atomiques" sont fait à l'aide de MD5.

\subsection{Comment combiner des hash}

\subsection{Comment hasher les feuilles}
