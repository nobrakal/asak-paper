\documentclass[a4paper]{easychair}
\usepackage[utf8]{inputenc}
\usepackage[T1]{fontenc}
\usepackage[french]{babel}
\usepackage{minted}
\usepackage{amsthm}
\usepackage{hyperref}
\usepackage{amsfonts}
\usepackage{multicol}
\usepackage{tikz}
\usetikzlibrary{arrows}
\usepackage{macros}
\pagestyle{empty}

\title{Détection de définitions OCaml similaires \\
(ou comment ne plus voir double à dos de chameau)}

\author{Alexandre Moine et Yann Régis-Gianas}

\institute{
Université de Paris (UMR-CNRS 7126) - INRIA PI.R2
}

\pagenumbering{gobble}
\authorrunning{Moine et Régis-Gianas}
\titlerunning{Détection de définitions OCaml similaires}
\begin{document}

\maketitle

\begin{abstract}
L'absence de redondance est souvent un gage de qualité pour un code
source. En effet, lorsqu'un fragment de code est répété, ses
imperfections -- pour ne pas dire ses erreurs -- le sont elles-aussi.

Seulement, il arrive parfois que l'on constate de la redondance dans
un grand corpus de code, typiquement quand ce corpus a été construit
par des développeurs ne communiquant peu ou pas du tout entre
eux. Deux instances de cette situation nous intéressent
particulièrement : l'ensemble des codes sources des paquets OPAM et
l'ensemble des copies d'étudiants répondant tous aux mêmes questions
de programmation. Comment partitionner leurs définitions en fonction
de leur ``similarité''?

Dans cet article, nous proposons un outil de partitionnement
automatique d'un ensemble de définitions écrites en OCaml. Cet outil
s'appuie sur une fonction dédiée de prise d'empreintes des arbres de
syntaxe du langage intermédiaire {\LambdaCode} ainsi que sur un
algorithme de classification hiérarchique classique que nous avons
adapté à notre usage.

Cet outil prend la forme d'une bibliothèque nommée {\Asak} disponible
sur {\Opam}. Nous l'avons utilisée d'une part pour partitionner
automatiquement les réponses d'étudiants qui apprennent OCaml en
utilisant la plateforme LearnOCaml, et d'autre part, pour détecter des
redondances sur l'ensemble des codes sources des paquets~{\Opam}
disponibles aujourd'hui. Nous évaluons les résultats obtenus et
formulons quelques limites de notre approche.

\end{abstract}

\section{Introduction}
%!TEX root = root.tex

\subsection{Contributions}
\begin{itemize}
\item Hash $ \lambda$
\item Clustering
\item Asak
\item LearnOCaml
\item Ananlyse des paquets opam
\end{itemize}

\section{Vue d'ensemble de l'approche}
\label{sec:overview}
Dans cette section, nous décrivons l'architecture de notre outil sur
un corpus d'exemple. Nous justifions nos choix de conception par la
même occasion.

\yrg{Il faut dire à un moyen que l'on compare les définitions deux à deux
et non pas tous les sous-termes deux à deux.}

\paragraph{Que produit {\Asak}?}
La figure~\ref{fig:example:sources} contient les cinq
exemples de définitions {\OCaml} qui forment un corpus jouet
qui va nous servir à illustrer le fonctionnement d'{\Asak}.
Le lecteur aura reconnu d'un coup d'{\oe}il la fonction classique qui
renvoie la liste mirroir d'une liste prise en argument et l'enseignant
aura reconnnu des réponses typiques d'étudiants apprenti-programmeurs
fonctionnels.

\begin{figure}

\begin{ocaml}
(* Code 1 *)
let rec rev l = match l with
    [] -> []
  |t::q -> rev q@[t]

(* Code 2 *)
let rec rev l =
  match l with
  | [] -> []
  | a::t -> (rev t)@[a]

(* Code 3 *)
let rec rev l = match l with
    [] | [_] -> l
  | t::q -> rev q@[t];;

(* Code 4 *)
let rev l=
  let rec rev_aux acc l=
    match l with
    |[]->acc
    |t::q->rev_aux (t::acc) q
  in rev_aux [] l

(* Code 5 *)
let rev l =
  match l with
    [] -> []
  |a::q -> let rec rev2 x y = match y with
        [] -> x
      |b::z -> rev2 (b::x) z in rev2 [] l

(* Code 6 *)
let rev l =
  List.fold_left (fun acc x -> x :: acc) [] l
\end{ocaml}

\label{fig:example:sources}
\caption{Un corpus jouet pour illustrer notre algorithme.}
\end{figure}

Sur ce corpus, notre outil produit le dendrogramme suivant~:

\begin{center}
\begin{tikzpicture}[sloped, scale=0.5]
\node (a)  at (-6,0)   {1};
\node (b)  at (-5,0)   {2};
\node (ab) at (-5.5,1) {};
\node (c) at (-3,0) {3};
\node (abc) at (-4, 2) {};
\node (d) at (-2,0) {4};
\node (e) at (-1,0) {5};
\node (de) at (-1.5,2) {};
\node (abcde) at (-3.5, 3) {};
\node (f) at (0,0) {6};
\node (all) at (-1, 4) {};
\node (root) at (-1,5) {};

\draw  (a) |- (ab.center);
\draw  (b) |- (ab.center);
\draw  (ab) |- (abc.center);
\draw  (c)  |- (abc.center);
\draw  (abc) |- (abcde.center);
\draw  (de)  |- (abcde.center);
\draw  (d)  |- (de.center);
\draw  (e)  |- (de.center);

\draw  (abcde) |- (all.center);
\draw  (f)     |- (all.center);

\draw (all)      |- (root.center);

\draw[->,-triangle 60] (-7,0) -- node[above]{dissimilarité} (-7,6);
\end{tikzpicture}
\end{center}

Un dendrogramme est un arbre binaire dont les n{\oe}uds représentent
des classes d'individus. Les feuilles de ce dendrogramme sont les
différentes versions de \texttt{rev}. Un dendrogramme fournit un
partitionnement hiérarchique d'un ensemble d'invididus: en le
parcourant d'une feuille vers sa racine, on découvre des classes
d'invididus de plus en plus dissimilaires à cette feuille ; en le
parcourant de sa racine vers ses feuilles, on découvre des séparations
successives en deux classes d'un ensemble d'individus maximisant
la dissimilarité entre les membres des deux classes. \yrg{Formulation
un peu lourde et peut-être incorrecte. Alexandre, tu peux m'aider?} \am{Peut-être peut-on expliquer en utilisant la construction même de l'arbre: Choisissant pour distance entre deux arbres le maximum des dissimilarités entre leurs feuilles, on construit le dendrogramme en commençant avec un collection de feuilles puis en agglomérant les arbres les plus proches. Ainsi, plus deux feuilles ont un parent proche, plus elles sont similaires comparé aux autres feuilles.}

Dans le cas de ce corpus, le regroupement des définitions~$1$ et~$2$
est très naturel puisque ces deux définitions sont équivalentes à
quelques détails purement textuels près : la présence du \iocaml{|},
d'un retour à la ligne et d'un renommage. La définition~$3$ est proche
de cette première classe: elle en diffère seulement à cause d'un cas
d'analyse supplémentaire. Le regroupement des définitions~$4$ et~$5$
est assez naturel puisqu'elles suivent la même (et bonne!) stratégie
qui consiste à accumuler le résultat dans l'argument d'une fonction
auxiliaire interne. Notons que ces deux définitions ne sont pas
syntaxiquement équivalence même si elles utilisent les mêmes ``ingrédients'' :
la définition~$5$ effectue une analyse par cas supplémentaire pour
traiter le cas de la liste vide de façon spécifique. Les deux classes~$(1-2)-3$
et~$4-5$ partagent le fait d'être des définitions récursives procédant par
analyse de cas : ces individus sont clairement séparés de la définition~$6$
qui s'appuie sur une spécialisation de la fonction~\iocaml{List.fold_left}.

\paragraph{Comment procède {\Asak}?}
Ce partitionnement semble somme toute assez naturel mais comment notre
outil l'a-t-il obtenu? Comme nous l'avons déjà écrit dans
l'introduction, le traitement de notre outil se décompose
essentiellement en trois étapes: (i) les programmes sources sont
normalisés pour ignorer les détails syntaxiques que nous jugeons
inessentiels ; (ii) on calcule une empreinte pour caractériser la
structure et les ingrédients principaux des programmes normalisés ;
(iii) on applique un algorithme de partitionnement hiérarchique qui
s'appuie sur les empreintes.
\yrg{Rajouter une figure?}

\paragraph{Comment les définitions sont-elles normalisées?}
L'analyse syntaxique est la solution canonique pour éliminer les
artefacts textuels et ne garder que la structure d'un code source.
Notre outil travaille donc sur des arbres de syntaxe abstraits.  On
pourrait opposer à ce choix de conception le fait que considérer le
texte des programmes permettrait de séparer des programmes dont les
arbres de syntaxe sont égaux. Cependant, un tel point de vue sépare
trop de programmes.

Une fois que l'on a décidé de travailler sur un langage d'arbres
distincts de la syntaxe concrète, il faut choisir ce nouveau langage.
On aurait pu choisir de traduire les termes sources vers un langage
conçu pour l'occasion mais ce serait beaucoup de travail. Il existe
heureusement le langage intermédiaire {\LambdaCode} dans le
compilateur et nos expérimentations nous portent à croire qu'il se
place au bon niveau d'abstraction pour capturer la structure
calculatoire du programme source.

Ce langage sera présenté précisément dans la section~\ref{sec:lambda}
mais nous pouvons d'ores et déjà donner la traduction du corpus jouet
dans la figure~\ref{fig:lambda-corpus}. Pour réaliser cette
traduction, nous réutilisons la partie avant du compilateur~{\OCaml}
et nous effectuons un post-traitement qui normalise encore un peu plus
les termes. Les détails de cette traduction seront décrits dans la
section~\ref{sec:lambda-normalization}. Sur nos exemples, on peut déjà
remarquer que les définitions~$1$ et~$2$ sont identiques une fois
normalisées. On note aussi que la définition~$3$ normalisée partage
des sous-termes avec les définitions~$1$ et $2$ normalisées. Des
remarques similaires s'appliquent aux autres définitions.

\begin{figure}
\begin{lisp}
Définition 1:
(function l/88
  (if l/88
    (apply (field 36 (global Stdlib!)) (apply rev/87 (field 1 l/88))
      (makeblock 0 (field 0 l/88) 0a))
    0a))

Définition 2:
(function l/88
  (if l/88
    (apply (field 36 (global Stdlib!)) (apply rev/87 (field 1 l/88))
      (makeblock 0 (field 0 l/88) 0a))
    0a))

Définition 3:
(function l/88
  (catch
    (if l/88
      (if (field 1 l/88)
        (apply (field 36 (global Stdlib!)) (apply rev/87 (field 1 l/88))
          (makeblock 0 (field 0 l/88) 0a))
        (exit 12))
      (exit 12))
   with (12) l/88))

Définition 4:
(function l/88
  (letrec
    (rev_aux/89
       (function acc/90 l/91
         (if l/91
           (apply rev_aux/89 (makeblock 0 (field 0 l/91) acc/90)
             (field 1 l/91))
           acc/90)))
    (apply rev_aux/89 0a l/88)))

Définition 5:
(function l/88
  (if l/88
    (letrec
      (rev2/91
         (function x/92 y/93
           (if y/93
             (apply rev2/91 (makeblock 0 (field 0 y/93) x/92) (field 1 y/93))
             x/92)))
      (apply rev2/91 0a l/88))
    0a))

Définition 6:
(function l/88
  (apply (field 20 (global Stdlib__list!))
    (function acc/146 x/147 (makeblock 0 x/147 acc/146)) 0a l/88))
\end{lisp}
\caption{Traduction du corpus dans le code {\LambdaCode} du compilateur~\OCaml.}
\label{fig:lambda-corpus}
\end{figure}

\paragraph{Pourquoi calcule-t-on des empreintes?}

Comparer deux arbres en itérant sur leurs structures respectives a un
coût proportionnel à leur taille. Par ailleurs, la notion de
comparaison qui nous intéresse doit idéalement savoir comparer
l'ensemble des sous-arbres des deux arbres pour déterminer à quel
point ils sont construits avec des ingrédients similaires, i.e.  à
quel point ils ont du code en commun. L'ordre d'apparition des
sous-arbres n'est donc pas forcément important : bien entendu, deux
arbres utilisant les mêmes sous-arbres dans le même ordre seront
très similaires mais utiliser les mêmes sous-arbres dans un ordre
différent est aussi une forme de similarité non négligeable même
si elle est un peu moins forte.

Après ces remarques, l'implémentation d'une fonction d'évaluation de
la similarité entre deux termes $\LambdaCode$ semble difficile. Nous
introduisons les empreintes d'arbres pour simplifier cette
implémentation et aussi pour la rendre efficace. Les empreintes sont
des ensembles de clés de hachages des sous-termes (suffisamment gros)
du terme $\LambdaCode$. L'idée importante de cette notion d'empreinte
est de prendre en compte l'ordre relatif des sous-termes dans le
calcul de la clé de hachage mais de voir l'empreinte comme un ensemble
de clés pour maintenir une certaine proximité entre les termes qui
utilisent les mêmes sous-termes, mais dans un ordre différent. Ainsi,
les deux programmes suivants:
\begin{ocaml}
let f () = e1; e2
let g () = e2; e1
\end{ocaml}
\noindent ont pour empreintes:
\[
\begin{array}{rcl}
E(\iocaml{f}) &=& \{ H(\iocaml{e1; e2}), H(\iocaml{e1}), H(\iocaml{e2}) \} \\
E(\iocaml{g}) &=& \{ H(\iocaml{e2; e1}), H(\iocaml{e1}), H(\iocaml{e2}) \}
\end{array}
\]
\noindent où $E(t)$ est l'empreinte de $t$ et $H(t)$ est la clé de hachage de $t$.

Ces deux empreintes sont distinctes mais partagent deux clés de
hachage. Ce partage témoigne du fait qu'elles ``utilisent les mêmes
ingrédients''. Les empreintes calculées pour les définitions de notre
corpus jouet se trouvent dans la figure~\ref{fig:hash} Nous décrivons
la définition précise de cette prise d'empreintes dans la
section~\ref{sec:hash}.

\begin{figure}
TODO
\caption{Les empreintes des définitions de notre corpus jouet.}
\label{fig:hash}
\end{figure}

\paragraph{Comment le partitionnement hiérarchique est-il effectué?}


\section{Prise d'empreintes}
\label{sec:hash}
%!TEX root = root.tex

Dans cette section, nous formalisons la prise d'empreintes des
définitions. Dans la section~\ref{sec:lambda}, nous donnons la syntaxe
du langage~$\LambdaCode$ ainsi que les deux normalisations que nous
appliquons à ses termes. Dans la section~\ref{sec:fingerprint},
nous donnons la définition précise de notre fonction de prise
d'empreintes.

\subsection{$\LambdaCode$ normalisé}
\label{sec:lambda}

\subsubsection{Présentation de $\LambdaCode$}

\begin{figure}

\[\small
\begin{array}{rclr}
\term
  & ::= & \var & \legend{Variable} \\
  & |   & \const & \legend{Constante} \\
  & |   & \apply\term{\many\term} & \legend{Application} \\
  & |   & \lam{\many\var}\term & \legend{Abstraction} \\
  & |   & \tlet\binding\term & \legend{Définition locale} \\
  & |   & \tletrec{\many\binding}\term & \legend{Définitions récursives} \\
  & |   & \apply\primitive{\many\term} & \legend{Appel d'une primitive} \\
  & |   & \switch\term{\many\branch}{\many\branch}{\term}
        & \legend{Branchement} \\
  & |   & \staticraise\nat{\many{\term}} & \legend{Saut local} \\
  & |   & \staticcatch{\term}\nat{\many{\var}}{\term} & \legend{Expression étiquetée} \\
  & |   & \trywith{\term}{x}{\term} & \legend{Expression étiquetée} \\
  & |   & \tifthenelse\term\term\term & \legend{Branchement conditionnel} \\
  & |   & \term ; \term & \legend{Séquencement} \\
  & |   & \twhile\term\term & \legend{Boucle non bornée} \\
  & |   & \tfor\var\term\term\term & \legend{Boucle bornée ascendante} \\
  & |   & \tfordown\var\term\term\term & \legend{Boucle bornée descendante} \\
  & |   & \tassign\var\term & \legend{Affectation} \\
  & |   & \tsend\term\term{\many\term} & \legend{Appel de méthode} \\
\binding & ::= & \var = \term & \legend{Définition} \\
\branch  & ::= & n \rightarrow \term & \legend{Branche} \\
\end{array}
\]
%% Les deux constructeurs suivants doivent être ignorés.
%%   | Levent of lambda * lambda_event
%%   | Lifused of Ident.t * lambda
\caption{La syntaxe de {\LambdaCode}
$\textrm{avec }
\nat \in \mathbb{N},
\var \in \mathcal{V},
\const \in \mathcal{C},
\primitive \in \mathcal{P}$.
}
\label{fig:lambda-syntax}
\end{figure}


La syntaxe du langage $\LambdaCode$ est donnée dans la
figure~\ref{fig:lambda-syntax}. Il s'agit d'un $\lambda$-calcul avec
quelques spécificités par rapport à celui que l'on trouve dans les
présentations à saveur plus théoriques. Tout d'abord, les fonctions ne
sont pas unaires : elles ont une arité potentiellement supérieure
à~$1$. Ensuite, on distingue les primitives des constantes : les
primitives sont nécessairement appliquées. Le langage contient un
fragment impératif permettant d'affecter des variables, d'itérer
\textit{via} des boucles \iocaml{for} et \iocaml{while}, et enfin de
détourner le flot du contrôle \textit{via} les différents mécanismes
de lancement et rattrapage d'exceptions. L'appel de méthode est la
construction qui nous rappelle qu'{\OCaml} est un aussi un langage à
objets. Pour finir, $\LambdaCode$ n'a pas d'analyse de motifs mais est
muni d'un branchement n-aire introduit par le mot-clé~\iocaml{switch}.

Par manque de place, nous ne donnons pas les règles de sémantique de
ce langage et nous laissons au lecteur le soin de réfléchir à ces
dernières. Par contre, pour se convaincre que l'on ne perd pas trop de
structure en calculant la redondance sur des termes de {\LambdaCode}
et non des termes {\OCaml}, il faut prendre le temps d'expliquer les
différences entre ces deux langages.

\paragraph{Absence de types}
{\LambdaCode} est un langage non typé. On n'y retrouve donc aucune
déclaration de types. Cette absence limite donc d'emblée le champ
d'application de {\Asak}: nous ne pouvons pas détecter de déclarations
de type redondantes. Cependant, cette limitation a un avantage :
lorsque l'on oublie les types, on se donne la possibilité de détecter
plus de redondances entre des programmes de types distincts mais
partageant la même structure. En revenant aux définitions des types
qui interviennent dans deux programmes ayant la même structure, on
peut espérer détecter indirectement des redondances entre les
définitions de types.
%
Un exemple d'une telle situation sera présentée
et discutée dans la section~\ref{sec:} \yrg{Donner l'endroit où on
discute de la fonction ``isomorphe'' à Option.map}.

\paragraph{Absence de modules et de classes}
Les constructions de seconde classe (comme les définitions de modules
et de classes d'objets) ont disparu dans le programme traduit
{\LambdaCode}. Cela limite notre capacité à détecter des modules
similaires ou des classes similaires. Comme pour les définitions de
type, nous pensons qu'en nous focalisant uniquement sur les aspects
calculatoires, nous pouvons détecter \textit{a posteriori} des
définitions de modules ou de classes similaires parce qu'elles
partagent des définitions similaires.

\paragraph{Absence d'analyse de motifs}
L'analyse de motifs d'{\OCaml} a été compilée en {\LambdaCode} sous la
forme d'arbres de décision, exprimés à l'aide d'expressions
conditionnelles et de branchements. Deux analyses de motifs distinctes
syntaxiquement en {\OCaml} peuvent être envoyées vers le même arbre de
décision et le même code~\LambdaCode: c'est par exemple le cas de deux
analyses à~$3$ branches sur le type \iocaml{color = Red | Black |
  White} car quelque soit l'ordre des branches de
l'analyse\footnote{En supposant qu'il n'y a pas de clause
  \iocaml{when} en jeu.}, celles-ci vont être traduites vers des
branchements à $3$ cas où l'ordre des cas correspond à l'ordre des
constructeurs de données dans la définition du type.

\subsubsection{$\alpha$-renommage}

Le nom des variables sera pris en compte lors du hash de la
structure. Il est donc important de s'assurer que les noms de
variables liées (par exemple lors d'une définition de fonction) ne
jouent aucun rôle.  C'est pourquoi nous faisons une itération sur
chaque arbre Lambda afin de renommer les variables liées selon leur
position.

\subsubsection{Inlining agressif}

L'arbre Lambda vient avec des informations sur les définitions
\verb|let| et notamment si elles peuvent être inliné (c'est-à-dire
remplacée par leur définition).

Nous inlinons donc toutes les définitions possibles.

\subsection{Formalisation}
\label{sec:fingerprint}

\paragraph{Notation}
On notera:

\begin{itemize}
	\item $H = \mathbb{N} \times 2^{128}$. Cela permet de représenter les couples poids/hash.
	\item $List_n(X)$ les listes de taille $n \in \mathbb{N}$ d'éléments de $X$
	\item $List_*(X) = \cup_{n\in\mathbb{N}}\ List_n(X)$
\end{itemize}

\subsection{La fonction de hash}

Nous suivons ici la méthodologie donnée par Chilowicz et
al. \cite{chilowicz:hal-00627811}: nous allons définir récursivement
une fonction de hash pour les arbres Lambda qui renverra aussi les
hashs des sous-arbre ainsi que leurs poids.

La fonction de hash est donc de type $hash : Lambda \to List_*(H)$.

%todo Pourquoi ?
Les hashs "atomiques" sont fait à l'aide de MD5.

\subsection{Comment combiner des hash}

\subsection{Comment hasher les feuilles}


\section{Partitionnement de corpus dirigé par la similarité}
\label{sec:clustering}
%!TEX root = root.tex

Une première approche est de partitionner une liste de code selon la fonction de hash: deux codes sont équivalents s'ils partagent exactement la même liste de hash. Cependant, cela ne permet pas de détecter que deux codes sont "proches". Par exemple, avec

\begin{minted}{OCaml}
let id1 x = print_endline "debug"; x
let id2 x = x
\end{minted}

\verb|id1| et \verb|id2| n'ont pas la même liste de hash (et nous ne voulons pas qu'ils l'aient), mais nous voudrions quand même détecter que les deux codes se ressemblent. Nous sommes donc revenu à généraliser la partition des codes selon leur hash en une partition selon une mesure de similarité entre codes: il s'agit de faire faire un clustering.

De nombreuses méthodes de clustering existent. Ne connaissant pas a priori le nombre de classes recherchées, nous avons décidé de nous orienter vers un clustering hiérarchique agglomératif: on regroupe les deux classes les plus proches (typiquement dans un arbre appelé dendrogramme) puis on répète l'opération jusqu'à n'avoir qu'une seule grande classe.

Nous introduisons cependant une légère variante: notre algorithme retourne une liste de classe infiniment distante. En effet, nous ne voulons pas regrouper tous les codes en une unique classe d'équivalence.

Nous devons donc définir deux choses:
\begin{itemize}
\item La distance entre deux listes de hash.
\item La distance entre deux classes de liste de hash.
\end{itemize}

\subsection{Distance entre deux listes de hash}
Il nous faut définir une distance, ou plus précisément une mesure de dissimilarité entre deux listes de hash.

\begin{align*}
d &: List_n(H) \times List_m(H) \to \mathbb{N} \cup \{\infty\} \\
d (X,Y) &=
\begin{cases}
	\infty & \text{si $X \cap Y = \emptyset$} \\
	\sum\limits_{h \in X \Delta Y} (fst\ h) & \text{sinon}
\end{cases}
\end{align*}

%todo Proprement définir la notion de différence symmétrique pour deux listes
Où $X \Delta Y$ est la différence symétrique entre deux listes $X$ et $Y$.

Si deux listes ne partagent pas un seul hash, elles sont donc infiniment distantes. Sinon, on somme le poids des arbres qu'elles n'ont pas en commun.

Il est facile de voir que $d$ est une fonction de séparation ($\forall X,Y,\ d(X,Y) = 0 \iff X = Y$) et symétrique ($\forall X,Y,\ d(X,Y) = d(Y,X)$). Elle ne respecte cependant pas l'inégalité triangulaire et n'est pas donc une distance.


\subsection{Distance entre deux classes de liste de hash}
Il existe ici plusieurs approches.
%todo Pourquoi celle-ci ?
Nous avons décidé d'étendre $d$ à:

\begin{align*}
d &: \mathcal{P}(List_*(H)) \times \mathcal{P}(List_*(H)) \to \mathbb{N} \cup \{\infty\}\\
d(\alpha,\beta) &= \max\limits_{X \in \alpha, Y \in \beta} d(X,Y)
\end{align*}

Cette approche est classique et donne lieu à un \emph{complete linkage clustering}.

\subsection{Algorithme}
L'algorithme de clustering est le suivant:

\begin{minted}{OCaml}
type distance = Infinity | Regular of int
type hash
type cluster

(* Groupe les noms ayant le même hash *)
val group_by_hash : (name * hash) list -> (hash * name list) list
(* Crée un cluster singleton *)
val singleton : (hash * name list) -> cluster
(* Renvoie les deux clusters les plus proches selon d, ainsi que leur distance *)
val get_closest_with_d : cluster list -> (distance * (cluster * cluster))
(* Fusionne deux clusters *)
val merge : cluster -> cluster -> cluster list -> cluster list

(* Renvoie une liste de cluster deux à deux infiniment distants *)
let rec run_clustering xs =
  match xs with
  | [] | [_] -> xs
  | _ ->
    let (p, (u,v)) = get_closest_with_d xs in
    match p with
    | Infinity -> xs
    | Regular p -> run_clustering (merge u v xs)

let cluster (hash_list : (name * hash) list) : cluster list =
  run_clustering @@
    List.map singleton @@
      group_by_hash hash_list
\end{minted}

\section{Partitionnement de code étudiant}
\label{sec:partition}
%!TEX root = root.tex

\subsection{Motivation}
Une centaine d'étudiants de troisième année suivent le cours de programmation fonctionnelle. Ils ont 2 heures de travaux pratiques par semaine. Comment le professeur peut-il (humainement) analyser les réponses aux exercices de la semaine ?

L'utilisation de {\LearnOCaml} est déjà d'une grande aide: la correction automatique des exercices permet de voir les pourcentages de réussite de chaque étudiant. Cela pose cependant un problème: environ 80\% des étudiants ont eu tous les points ! Il s'agit maintenant de comprendre \emph{comment} les étudiants ont répondu; c'est-à-dire de comparer leur code.

Nous avons donc intégré {\Asak} à {\LearnOCaml} afin de classifier les codes des étudiants. En observant les représentants des classes obtenues et leur taille, le professeur peut ainsi se faire une idée en très peu de temps de la façon dont les élève ont répondu, ainsi qu'identifier les élèves qui ont fournis une réponse tout à fait distincte des autres et nécessitent peut-être plus d'attention.

\subsection{Approche utilisée}
Partant de l'hypothèse que les tests automatiques ont été bien écrits, nous classifions une première fois les codes selon la note qu'ils ont obtenu. En effet, nous ne voulons jamais identifier deux codes qui n'ont pas eu la même note, même s'ils sont similaire syntaxiquement. Cette première passe nous permet aussi d'avoir pour chaque classe une hypothèse, très forte, d'équivalence sémantique.
Cette hypothèse nous permet de raffiner la fonction de calcul d'empreinte afin d'identifier encore plus de codes.

\subsubsection{Amélioration de la fonction de calcul d'empreinte}
La fonction de calcul d'empreinte capture la forme de l'arbre et quelques plusieurs autres informations qui peuvent ici être réinterprétées.

\paragraph{La combinaison des empreintes}
Pour calculer l'empreinte d'un nœud, nous calculons en fait l'empreinte de la liste des empreintes de ses fils (avec l'ajout d'un sel propre au nœud en question). L'ordre des empreintes dans cette liste importe la plupart du temps. Mais ici, notre hypothèse  d'équivalence sémantique permet d'assurer que l'ordre n'importe pas. En effet, si c'était le cas, les codes auraient une sémantique différente. Nous trions donc toutes les listes d'empreintes selon un ordre arbitraire avant de les combiner.

\paragraph{L'empreinte des feuilles}
Nous avons déjà soulevé le problème du calcul des empreintes des feuilles de l'arbre en Section \am{TODO Insérer une référence à la section correspondante}. Dans ce contexte d'équivalence sémantique, nous pouvons supposer que dans deux arbres de même forme, les identifiants ne sont que des alias les uns des autres. En effet, s'il existait une différence sémantique entre deux identifiants, les deux codes n'auraient pas la même sémantique.
Nous avons donc choisi d'associer la même empreinte à tous les identifiants.

\subsection{Résultats}

Les exemples de la figure \ref{fig:hash} sont tirés d'un corpus plus gros: celui des réponses des étudiants de troisième année à la question "implémentez la fonction \verb|rev|".

\am{Quoi mettre ? Une capture d'écran ? Une description des classes obtenues ?}


\section{Détection de redondance dans les paquets {\Opam}}
\label{sec:redundancy}
%!TEX root = root.tex

La redondance de code est un des fléaux de l'informatique moderne. Elle entraine inévitablement une grande difficulté de maintien (si l'on change une définition, il faudrait changer tous ses clones), de compréhension (nous ne dénombrons pas moins de 30 noms différents pour la célèbre fonction \verb|Option.map| dans la moitié des paquets \Opam) et des pertes de temps énorme consacrés à redéfinir des fonctions usuelles (nous dénombrons 142 implémentations de \verb|Option.map| dans la moitié des paquets \Opam).

Comme le lecteur avisé l'a compris, nous avons lancé {\Asak} sur l'ensemble des paquets {\Opam}.

\subsection{Approche utilisée}

Un premier obstacle est la construction de l'entièreté des paquets {\Opam} afin d'obtenir les expressions lambda des tous leurs fichiers sources.

Nous avons choisi de travailler à l'aide d'un compilateur modifié qui normalise et exporte les arbres lambdas tout ce qu'il compile vers un fichier. Il nous a ensuite suffit d'essayer d'installer chaque paquet {\Opam} à l'aide de ce compilateur. Nous avons donc développé un outil très léger permettant de lancer toutes ces opérations dans un switch {\Opam}.

\subsection{Analyse des résultats}

Nous avons réussi à compiler 1250 paquets sur les 2428 paquets que compte à ce jour le dépôt {\Opam}. Tous les paquets n'ont pas pu être installé car certains:

\begin{itemize}
\item Ne compilent pas avec {\OCaml} 4.08.01
\item Entrent en conflit avec {\OCaml} 4.08.01
\item Dépendent de librairies C non-installées
\end{itemize}

Sur ces 1250 paquets, nous avons extraits 356 703 définitions toplevel strictement différentes (c'est-à-dire sans compter les définitions d'un même paquet de deux versions différentes qui produisent la même empreinte).



\section{Limitations}
\label{sec:limitations}
%!TEX root = root.tex

\subsection{Le presque-sucre syntaxique}

Nous programmons souvent en utilisant des constructions semblant
similaires mais qui sont bel et bien des constructions sémantiquement
différentes. Cependant {\Asak} est par définition très sensible à la
forme de l'arbre {\LambdaCode} et ce "presque-sucre syntaxique" fait
différer les arbres de codes pourtant très proches. Il en résulte
qu'{\Asak} n'est \emph{pas} adapté à la détection de plagiat.
\yrg{Je ne suis pas sûr de te suivre. Tu as un exemple concret en tête?}

\paragraph{Les définitions mutuellement récursives}

Il est par exemple très courant de définir des fonctions mutuellement
récursives alors qu'elles ne le sont pas. Le compilateur n'étant pas
capable de différencier ces cas, les arbres engendrés seront très
différents selon si des définitions successives ont étés faites via
\verb|let| ou \verb|let rec|.

\paragraph{L'inlining}

Nous remplaçons souvent mentalement les variables par leur
définition. Il s'agit cependant d'une opération complexe qui affecte
souvent la sémantique du programme et que le compilateur se risque
rarement à faire. Il en résulte que deux codes ayant exactement la
même sémantique, l'un ayant une multitude de définitions factorisées
et l'autre non engendrent des arbres très différents.


\section{Travaux connexes}
\label{sec:related-work}
\input{related-work}

\section{Conclusion et travaux futurs}
\label{sec:conclusion}
Dans cet article, nous avons présenté l'approche suivie par~{\Asak} pour
la détection de clones de programmes~{\OCaml} ainsi que des résultats
préliminaires concernant l'application de cet outil à l'analyse du corpus de
paquets {\Opam}.
%
Pour pallier aux limitations que nous avons explicitées, nous allons ...
%
Enfin, en intégrant {\Asak} a un outil comme {\Merlin}, nous allons proposer
au programmeur un outil pour éviter d'introduire de la redondance dans
les programmes~{\OCaml}.

Merci la Fondation OCaml.


\bibliographystyle{plain}
\bibliography{publications}

\end{document}
