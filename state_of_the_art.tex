%!TEX root = root.tex

La comparaison de code est un sujet énormément étudié.
Différentes analyses détaillées des techniques existantes ont déjà été faites par Roy et al. \cite{ComparisonAndEvaluation} et Gautam et al. \cite{variousCodeClone}.
On peut grouper les différentes approches en quatre grandes familles qui utilisent chacune plus ou moins la sémantique du langage dans lequel le code est écrit :
\\
\begin{itemize}
\item[\emph{Approche textuelle:}] Il s'agit ici de comparer directement les chaines de caractères composant le code, avec le plus souvent un pré-traitement visant à supprimer les espaces inutiles et les commentaires. Cette approche est la seule indépendante du langage (à l'exception de la phase de pré-traitement). Elle a été mise en œuvre dans des outils comme Duploc \cite{ALanguageIndependent}.

\item[\emph{Approche lexicale:}] D'autre outils choisissent de travailler sur la séquence de tokens correspondant au code. Ces outils deviennent donc dépendant d'un langage mais permettent de résoudre certains problèmes de l'approche purement textuelle (on est par exemple capable d'identifier deux codes égaux à $\alpha$-renommage près).
Cette méthode est implémentée dans des outils comme CC-Finder \cite{ccfinder}, CP-Miner \cite{cpminer} et DUP \cite{dup}.

\item[\emph{Approche syntaxique:}] Une autre approche est d'utiliser directement l'arbre de syntaxe abstrait correspondant au code. Elle est donc dépendante du langage et permet de résoudre certains problèmes de l'approche lexicale. Cette méthode a été popularisée avec CloneDr \cite{CloneDetectionUsingAST} et utilisée plus récemment par Deckard \cite{Deckard}.

\item[\emph{Approche sémantique:}] Enfin, on peut aussi utiliser les graphes de dépendance \cite{PdgOrigin} du programme. Ces derniers permettent d'introduire beaucoup de sémantique dans la détection de clone mais la technique repose sur la recherche de sous-graphes identiques maximaux, problème qui est NP-complet. Des approximations en temps polynomial permettent néanmoins d'obtenir de bons résultats, comme l'a montré Krinke \cite{PDG}.
\end{itemize}
